% Awesome CV LaTeX Template
%
% This template has been downloaded from:
% https://github.com/huajh/huajh-awesome-latex-cv
%
% Author:
% Junhao Hua


%Section: Work Experience at the top
\sectionTitle{项目经历}{\faCode}
\begin{experiences}
 \experience
    { 2019年7月 }  { 智能辅助诊断系统 }{ 后端 }{ 第三方数据库接口 }
    { 2019年10月 } {
                      \begin{itemize}
                        \item 为系统提供访问医院不同类型数据库的统一接口
                        \item 实现了基于视图和存储过程访问MySQL、SQLServer、Oracle的统一接口
                        \item 抽象出Column、Condition和Parameter
                        \item 抽象出不同类型数据库查询接口,可以接受SQL串并以Column返回
                        \item 根据医院提供的配置自动拼接生成不同数据库类型的SQL串,并交给查询接口
                        \item \emph{已在医院投入使用}
                      \end{itemize}
                      \vspace{2pt}
                    }
                    {C++, 数据库}
  \experience
    { 2019年7月 }   { 智能辅助诊断系统 }{ 后端自动化Smoke工具 }{ 设计开发 }
    { 2019年10月 }  {
                      \begin{itemize}
                        \item 功能:快速测试系统后端功能是否符合预期
                        \item 采用了前后端微服务的设计,把测试功能作为一个接口可以通过http调用和控制
                        \item 每次收到Json请求,会解析Json并作为参数创建一个Task到Task队列,执行进行会依次从队列里面获取Task并执行测试
                        \item 把中间抽象为Process,一次Smoke过程可以看成是不同Process的组合,并提供工厂类生成不同类型Process
                        \item Python和Sanic框架作为后端提供微服务接口
                        \item 用HTML和JS写了一个简易的前端,可以方便以前端浏览器的方式(可在浏览器控制测试参数)触发测试
                        \item 与TFS结合,TFS出包会自动发送json请求开启对该包的自动smoke测试
                        \item 完成自动化Smoke测试会自动发邮件给相关的人
                        \item \emph{已在后端内部长时间稳定使用}
                      \end{itemize}
                      \vspace{2pt}
                    }
                    {Python, js}
  \experience
    { 2019年7月 }   { 智能辅助诊断系统 }{ Service端 }{ 前后端 }
    { 2019年10月 }  {
                      \begin{itemize}
                        \item Service端主要为智能辅助诊断系统的维护工程师提供方便的web工具
                        \item 主要功能包括系统的控制、各种配置文件的修改、控制系统自动升级、系统状态、系统使用情况
                        \item Python和Snaic框架编写后端,Vue.js前端
                        \item 从零开始构建该系统
                        \item \emph{已在医院投入使用}
                      \end{itemize}
                      \vspace{2pt}
                    }
                    {Python, Vue.js, Redis, Sanic}
  \emptySeparator
  \experience
    { 2019年11月 }  { 冠脉智能辅助诊断系统 }{ 算法 }{ 分段、中心线 }
    { 至今 }  {
                      \begin{itemize}
                        \item 冠脉的自动分段,对分割出的冠脉进行自动化分段,分成18段
                        \item 中心线的平滑和精准性,由于医生需要根据中心线的CPR完成诊断,对中心线的平滑性和精准性要求非常高
                        \item \href{https://www.zybuluo.com/zsh-o/note/1659960}{\color{blue}基于Snake方法的2D实验}
                      \end{itemize}
                      \vspace{2pt}
                    }
                    {C++, Python, Pytorch}
  
\end{experiences}
